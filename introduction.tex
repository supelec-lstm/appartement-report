

\chapter*{Introduction} 
\addcontentsline{toc}{chapter}{Introduction}
Nous sommes un groupe de six étudiants de CentraleSupélec, cursus Ingénieur Supélec. Dans le cadre d'un projet long s'étalant sur une année entière, nous avons été amenés à étudier les réseaux de neurones. Ce projet était encadré par deux enseignants-chercheurs de notre école, Mme Joanna Tomasik et M. Arpad Rimmel. L'objectif était d'étudier et d'implémenter différents réseaux de neurones. Ainsi, le projet était décomposé en plusieurs étapes. La première consistait à étudier le fonctionnement d'un réseaux de neurones simples et à l'implémenter. Puis, nous nous sommes intéressés aux réseaux de neurones récurrents offrant de nouvelles opportunités dans l'apprentissage. Deux algorithmes d'apprentissage furent ainsi étudiés : RTRL et BPTT. Enfin, nous sommes entrés dans le vif du sujet en nous penchant sur l'étude des LSTM. La fin du projet était destinée à appliquer les LSTM sur différents problèmes. Tout au long du projet, la démarche fut de rechercher les travaux déjà existants afin de s'approprier les différents aspects du problème et d'étudier les solutions proposées. Puis, nous implémentions nos propres réseaux de neurones sous Python afin de comparer leurs performances respectives. 
 
Tout au long du projet, plusieurs outils furent mis en place afin d'organiser et de faciliter le travail de groupe. GitHub (\url{https://github.com/supelec-lstm}) a été utilisé afin de partager le code entre les différents membres. Zotero permettait de tenir à disposition une bibliographie de tous les articles trouvés sur le sujet. Enfin, une réunion hebdomadaire avec nos professeurs et l'autre équipe permettait de faire le point sur les avancées de la semaine, de clarifier certaines incertitudes et de discuter de problèmes d'implémentation. 
 
Ce compte-rendu fait la synthèse du travail effectué lors de ce projet. On y trouvera des apports théoriques sur le fonctionnement des réseaux de neurones. De  plus, les implémentations choisies seront explicitées et les différentes méthodes d'apprentissage seront comparées. 
 