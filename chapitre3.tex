\chapter{Graphe de calculs}

\section{Motivations}

Nous avons vu précédemment que l'architecture du réseau de neurones peut être lourde autant du point de vue modélisation que du point de vue calcul. Pour motiver l'apparition des graphes de calcul prenons l'exemple d'un réseau de neurones à $m$ entrées, une couche cachée de $n$ neurones et $p$ sorties. Tous les neurones ont pour fonction d'activation une sigmoïde $\sigma$. Un tel réseau est représenté sur la figure \ref{reseau_3_5_3}.

\begin{figure}
\begin{center}
\begin{tikzpicture}[scale=0.5, line cap=round,line join=round,>=triangle 45,x=1.0cm,y=1.0cm]
\clip(-3.05,-9.53) rectangle (18.64,6.43);
\draw(2,2) circle (1cm);
\draw(2,-1) circle (1cm);
\draw(2,-4) circle (1cm);
\draw(10,2) circle (1cm);
\draw(10,-1) circle (1cm);
\draw(10,-4) circle (1cm);
\draw(6,2) circle (1cm);
\draw(6,-1) circle (1cm);
\draw(6,-4) circle (1cm);
\draw(6,5) circle (1cm);
\draw(6,-7) circle (1cm);
\draw [->] (3,2) -- (5,5);
\draw [->] (3,2) -- (5,2);
\draw [->] (3,2) -- (5,-1);
\draw [->] (3,-1) -- (5,5);
\draw [->] (3,-1) -- (5,2);
\draw [->] (3,-1) -- (5,-1);
\draw [->] (7,2) -- (9,2);
\draw [->] (7,2) -- (9,-1);
\draw [->] (7,-1) -- (9,2);
\draw [->] (7,-1) -- (9,-1);
\draw [->] (3,2) -- (5,-4);
\draw [->] (3,-1) -- (5,-4);
\draw [->] (3,-1) -- (5,-7);
\draw [->] (7,-4) -- (9,-4);
\draw [->] (7,-4) -- (9,-1);
\draw [->] (7,-4) -- (9,2);
\draw [->] (7,2) -- (9,-4);
\draw [->] (7,-1) -- (9,-4);
\draw [->] (7,-7) -- (9,-4);
\draw [->] (7,-7) -- (9,-1);
\draw [->] (7,-7) -- (9,2);
\draw [->] (7,5) -- (9,2);
\draw [->] (3,-4) -- (5,-7);
\draw [->] (7,5) -- (9,-1);
\draw [->] (7,5) -- (9,-4);
\draw [->] (3,2) -- (5,-7);
\draw [->] (3,-4) -- (5,-4);
\draw [->] (3,-4) -- (5,-1);
\draw [->] (3,-4) -- (5,2);
\draw [->] (3,-4) -- (5,5);
\draw (1.5,2.5) node[anchor=north west] {$ x_1 $};
\draw (1.5,-0.5) node[anchor=north west] {$ x_2 $};
\draw (1.5,-3.5) node[anchor=north west] {$ x_3 $};
\draw (9.5,2.5) node[anchor=north west] {$ y_1 $};
\draw (9.5,-0.5) node[anchor=north west] {$ y_2 $};
\draw (9.5,-3.5) node[anchor=north west] {$ y_3 $};
\end{tikzpicture}
\end{center}
\caption{Réseau de neurones à 3 entrées, 5 neurones cachés et 3 neurones de sorties}
\label{reseau_3_5_3}
\end{figure}

Nous pourrions modéliser ce réseau de neurone en définissant $n+p$ neurones ayant chacun son vecteur poids mais une façon équivalente de modéliser ce système est de rassembler les poids des deux couches dans deux matrices $W_1 \in \mathbb{R}^{m \times n}$ et $W_2 \in \mathbb{R}^{n \times p}$. En notant $x \in \mathbb{R}^m$ le vecteur ligne des entrées du réseau, on a alors que la sortie est :
\begin{equation}
y = \sigma(\sigma(xW_1)W_2)
\label{eq_3_5_3}
\end{equation}
où la fonction $\sigma$ s'applique terme à terme. 

Sur la figure \ref{reseau_3_5_3}, à droite, est représentée cette formule de manière graphique. Un tel graphe sera appelé graphe de calculs.

Cette modélisation matricielle a de nombreux avantages. Tout d'abord, elle est plus compacte et simple à manipuler et à visualiser. Puis, elle a un avantage certain pour notre implémentation en Python car elle permet d'utiliser pleinement la librairie d'algèbre linéaire \texttt{numpy} ce qui accélerera grandement les calculs. Enfin, il est facile d'étendre la formule précédente au calcul de la sortie de plusieurs vecteurs d'entrées. En effet si $X \in \mathbb{R}^{N \times m}$ est une matrice contenant $N$ vecteurs d'entrées alors la sortie du réseau s'écrit :
$$Y = \sigma(\sigma(XW_1)W_2)$$

On aurait donc envie de s'affranchir de la structure du neurone pour seulement modéliser les opérations mathématiques effectuées par le réseau.

\section{Définitions}

\begin{definition}
Un graphe de calcul est un graphe représentant une fonction mathématique. Un n\oe{}ud de ce graphe représente soit une une variable, des poids ou une opération mathématique. Formellement, nous noterons $\mathcal{G}(V, A, V_{var}, (V_{op}, f), (V_{poids}, \theta))$ le graphe $\mathcal{G}$ dont les n\oe{}uds sont $V$, les arêtes $A$, $V_{var} \subset V$ les n\oe{}uds réprésentant une variable, $V_{op} \subset V$ ceux représentant une opération et $f$ les opérations en question, et $V_{poids} \subset V$ ceux représentant des poids et $\theta$ les poids associés.
\end{definition}

Dans cette définition, nous considérons les variables comme les arguments du graphe de calculs. En ce qui concerne, les réseaux de neurones, les entrées seront un n\oe{}ud représentant une variable. Les poids sont les paramètres de la fonction. Ce seront les poids des neurones par exemple. Nous n'imposons pour l'instant aucune limite aux opérations mathématiques pouvant être effectuées au sein des graphes de calculs.

Dans le graphe de calculs de la figure \ref{reseau_3_5_3}, le n\oe{}ud $x$ est une variable, les n\oe{}uds $W_1$ et $W_2$ sont des poids et les deux n\oe{}uds $\sigma$ représente l'opération sigmoïde appliquée terme à terme. Nous remarquerons que la sortie $y$ n'est pas explicitée par un n\oe{}ud. 

Les graphes de calculs offrent une très grande liberté. Il est possible de représenter avec ceux-ci une classe de fonctions bien plus grandes qu'avec des réseaux de neurones classiques.

Écrivons dès maintenant, l'algorithme de propagation :

\begin{algorithm} 
\begin{algorithmic}
\Procedure{$evaluer\_graphe$}{$\mathcal{G}(V, A, V_{var}, (V_{op}, f), (V_{poids}, \theta)), x$}
\Function{$evaluer\_noeud$}{$j$}
	\If{$déjàCalculé[j]$ est $faux$}
		\State $t \leftarrow (evaluer\_noeud(i), i \in Pa(j))$
		\State $y_j \leftarrow f_j(t)$
		\State $déjàCalculé[j] \leftarrow vrai$
	\EndIf
	\State \Return $y_j$
\EndFunction

\State Initialiser un tableau $dejaCalculé$ de longueur $|V|$ à $faux$.
\For{$i \in V_{var}$}
	\State $y_i \leftarrow x_i$
	\State $déjàCalculé[i] \leftarrow vrai$ 
\EndFor
\For{$j \in V$}
	\State $evaluer\_noeud(j)$ 
\EndFor
\EndProcedure
\end{algorithmic}
\caption{Algorithme d'évaluation d'un graphe de calculs.}
\label{propagation_memoisation}
\end{algorithm}

Nous remarquons que l'algorithme de propagation est très similaire à celui décrit pour les réseaux de neurones. En effet, les graphes de calculs en sont une généralisation.

Nous avons ainsi modéliser nos réseaux de manière plus efficace. Il reste cependant un problème à résoudre. Notre objectif est d'optimiser les paramètres de notre fonction. Pour cela, nous avons besoin de calculer la dérivée du coût $E$ par rapport aux poids avant de pouvoir utiliser l'algorithme de la descente du gradient.

\section{Dérivation automatique}

Afin de réaliser cela, nos n\oe{}uds représentant des opérations mathématiques ne seront pas seulement responsable de calculer une sortie mais aussi de propager le gradient. En effet prenons un n\oe{}ud représent une opération mathématique $f$ à $m$ entrées $x_1, ..., x_m$ et à $n$ sorties $y_1, ..., y_m$. La règle de la chaine, nous permet d'exprimer la dérivée du coût par rapport aux entrées en fonction du coût par rapport aux sorties du n\oe{}ud. En effet, on a :
\begin{equation}
\forall i \in \{1, ...,  m\}, \frac{\partial E}{\partial x_i} = \sum_{j=1}^{n}{\frac{\partial y_j}{\partial x_i}\frac{\partial E}{\partial y_j}}
\label{retropropagation_graphe}
\end{equation}

Rappelons que si $x_i \in \mathbb{R}^p$ et $y_j \in \mathbb{R}^q$ alors $\frac{\partial E}{\partial y_j} \in \mathbb{R}^q$ et $\frac{\partial y_j}{\partial x_i} = (\frac{\partial y_{j,l}}{\partial x_{i, k}})_{k, l} \in \mathbb{R}^{p \times q}$.

La formule \ref{retropropagation_graphe} nous permet de décrire un algorithme de rétropropagation du gradient très similaire à celui des réseaux de neurones :

\begin{algorithm} 
\begin{algorithmic}
\Procedure{$retropropager\_gradient$}{$\mathcal{G}(V, A, V_{var}, (V_{op}, f), (V_{poids}, \theta)), x, y$}
\Function{$calculer\_gradient$}{$j$}
	\If{$déjàCalculé_j$ est $faux$}
		\State $\frac{\partial E}{\partial y_j} \leftarrow \sum_{i \in Ch(j)}{calculer\_gradient(i)_j}$
		\State $\frac{\partial E}{\partial x_j} \leftarrow \frac{\partial E}{\partial y_j} g_j'(s_j) w_j$
		\State $\frac{\partial E}{\partial w_j} \leftarrow \frac{\partial E}{\partial y_j} g_j'(s_j) x_j$
		\State $déjàCalculé_j \leftarrow vrai$
	\EndIf
	\State \Return $\frac{\partial E}{\partial x_j}$
\EndFunction

\State Appeler $evaluer\_graphe(\mathcal{N}(V, A, V_{in}, V_{out}, f), x)$ et récupérer les entrées et sorties de chaque noeud
\State Initialiser un tableau $dejaCalculé$ de longueur $|V|$ à $faux$.
\For{$i \in V_{out}$}
	\State Calculer $\frac{\partial E}{\partial y_i}$
	\State $déjàCalculé_i \leftarrow vrai$ 
\EndFor
\For{$i \in V$}
	\State $calculer\_gradient(i)$ 
\EndFor
\EndProcedure
\end{algorithmic} 
\caption{Algorithme de rétropropagation du gradient dans un graphe de calculs.}
\label{propagation_memoisation2}
\end{algorithm}

\section{N\oe{}uds}


\section{Diagramme UML}

Là encore, l'architecture retenue contient 2 classes mères : Node et Graph, un graph étant constitué de plusieurs nodes qui connaissent chacune leurs parents et leurs enfants.

La différence principale avec l'architecture précédente est dans la modularité de la classe Node. Bien qu'elle joue un rôle similaire à celui de la classe Neuron dans le modèle Neuron/Network, elle est bien plus souple : une node peut représenter une entrée du graph, un vecteur de poids, une opération élémentaire (addition, soustraction, produit scalaire), ou une fonction (fonction d'activation d'un neurone ou fonction de coût du réseau). Une node prend pour entrée une matrice (ou 2 dans le cas des opérateurs élémentaires), et donne en sortie ses résultats sous la forme d'une matrice de taille adaptée. De même elle peut calculer le gradient par rapport à ses entrées étant donné celui de ses enfants toujours sous forme matricielle.  Cette capacité des nodes à gérer les matrices est primordiale car elle permet de créer un réseau ne possédant qu'une seule node d'entrée et de propager à travers lui tout un lot d'exemples en une seule fois, diminuant ainsi grandement le nombre d'appels objet en les remplaçant par des opérations matricielles. Les résultats obtenus avec cette architecture permettront par la suite de confirmer que les appels objets et non les opérations occupaient la majeure partie du temps de calcul dans l'implémentation présentée en chapitre 1.

\begin{figure}[!h]
\centering

\tikzset{every picture/.style={scale=0.6}}
\begin{tikzpicture}[line cap=round,line join=round,>=triangle 45,x=1.0cm,y=1.0cm, scale=0.5]
\clip(2.4143827288284365,2.1911915106170423) rectangle (28.896515457132203,16.893403649017483);
\draw(4.982,6.946) circle (1.5319647515527244cm);
\draw(10.46,6.946) circle (1.5319647515527248cm);
\draw(15.918356266000036,6.947742334000014) circle (1.5319647515527248cm);
\draw(21.396356266000033,6.947742334000014) circle (1.531964751552723cm);
\draw(26.874356266000035,6.947742334000014) circle (1.531964751552723cm);
\draw (4.683242626729719,7.418644715381643) node[anchor=north west] {$\mathcal{X}$};
\draw (10.001450227410324,7.781662299045852) node[anchor=north west] {$\times$};
\draw (15.119998157075615,7.509399111297696) node[anchor=north west] {$z=f(y)$};
\draw(15.911114126755379,11.67271742202839) circle (1.531964751552723cm);
\draw (14.575471781579308,10.141276592863207) node[anchor=north west] {sortie attendue};
\draw (14.212454197915102,5.222388334213182) node[anchor=north west] {fonction d'activation};
\draw (7.859646483791514,5.167935696663551) node[anchor=north west] {opération élémentaire};
\draw (4.10241449286699,5.1316339382971305) node[anchor=north west] {Entrée};
\draw (19.966282898992755,5.167935696663551) node[anchor=north west] {fonction de coût};
\draw (20.420054878573012,7.563851748847327) node[anchor=north west] {$c=h(z,y')$};
\draw (15.646373653388713,12.228627698932405) node[anchor=north west] {y'};
\draw (25.30264137885657,5.186086575846762) node[anchor=north west] {gradient constant};
\draw [->] (6.511832012049577,6.86519149235171) -- (8.930169512469373,6.865162635979358);
\draw [->] (12.00521159541307,6.880938976773743) -- (14.423549095832865,6.880910120401391);
\draw [->] (17.483211595413067,6.880938976773743) -- (19.901549095832866,6.880910120401391);
\draw [->] (22.92499832258255,6.977574257545156) -- (25.34333582300235,6.9775454011728035);
\draw(4.9814182705273335,11.336093947857117) circle (1.5319647515527193cm);
\draw (4.429130318164775,9.760108130015787) node[anchor=north west] {$poids$};
\draw (4.719544385096139,11.883760994451407) node[anchor=north west] {$w$};
\draw [->] (6.513382260277414,11.334566168750113) -- (8.905056271433413,6.865162935638539);
\draw [->] (17.442307496209068,11.624108108712399) -- (19.901549095832852,6.880910120401391);
\end{tikzpicture}
\caption{Schéma d'un graphe de calcul pour un neurone simple}
\label{graphe_de_calcul}
\end{figure}

Un graph commence par une node d'entrée, suivie d'un ensemble de nodes de calcul dont la dernière est qualifiée de node de sortie. Une ou plusieurs node(s) de fonction de coût prennent pour parent la node de sortie, une node d'entrée qu'on actualisera aux valeurs des sorties attendues lors de l'entrainement et une node de régularisation. Enfin on clôt le graph par une node qui possède un gradient unitaire constant et aucune sortie. Elle permettra d'initier la descente du gradient dans la pile des appels récursifs des nodes depuis les entrées vers la fin du graph. C'est le graph qui gère la propagation des exemples et la descente du gradient à travers les nodes mais son rôle est très limité de par la grande autonomie qui est donnée à celles-ci. Les fonctions de propagation et rétropropagation de chaque node sont en effet codée de sorte à ce qu'un unique appel de leur fonction de propagation (respectivement rétropropagation) sur la node d'entrée (respectivement de sortie) donne lieu à une cascade d'appel jusqu'à l'obtention du résultat voulu. Dans une propagation par exemple, on appelle la méthode « propager » sur la dernière node du réseau qui l’appellera elle-même sur ses parents pour calculer sa sortie, ce qui actualisera les sorties de toutes les nodes nécessaires au calcul du résultat initialement demandé.



\section{Résultats}

L'architecture en Graphe de Calcul donne de très bons résultats du point de vue de la rapidité de calcul. On observe par exemple une réduction d'un facteur 1000 du temps de calcul pour l'entrainement du réseau simple sur l'ensemble MNIST par rapport à l'implémentation "naïve" du chapitre 1. Ces performances sont atteintes grâce à la diminution du nombre d'appels objets, remplacés par des calculs matriciels sur des matrices de grande taille gérées par des bibliothèques efficaces.